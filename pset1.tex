\documentclass[11pt]{article}

% Packages for math. MUST LOAD BEFORE FONT PACKAGES
\usepackage{amsmath, amssymb, amsthm}


% Set font to Palatino
\usepackage{newpxmath}
\usepackage{newpxtext}
\linespread{1.05}         % Palatino needs more leading (space between lines)
\usepackage[T1]{fontenc}

\usepackage{microtype}
\usepackage{url}
\usepackage{tikz}
\usetikzlibrary{patterns}

\usepackage{color}
\usepackage{tcolorbox}



\usepackage[colorlinks=true,urlcolor=blue,linkcolor=blue]{hyperref}


% Page margins
\usepackage[margin=1in]{geometry}


% Theorem environments
\newtheorem{theorem}{Theorem}
\newtheorem{lemma}[theorem]{Lemma}
\newtheorem{proposition}[theorem]{Proposition}
\newtheorem{corollary}[theorem]{Corollary}
\newtheorem{fact}[theorem]{Fact}
\newtheorem{definition}[theorem]{Definition}
\newtheorem{example}[theorem]{Example}
\newtheorem{remark}[theorem]{Remark}
\theoremstyle{definition}
\newtheorem{problem}[theorem]{Problem}
\newtheorem*{question}{Question}
\newtheorem{exercise}[theorem]{Exercise}

% Delimiter macros
\newcommand{\paren}[1]{\left( #1 \right)}
\newcommand{\brac}[1]{\left[ #1 \right]}
\newcommand{\iprod}[1]{\langle #1 \rangle}
\newcommand{\abs}[1]{\left| #1 \right|}
\newcommand{\norm}[1]{\left\| #1 \right\|}

% Macros for real and natural numbers
\newcommand{\R}{\mathbb{R}} % Real numbers
\newcommand{\N}{\mathbb{N}} % Natural numbers

\renewcommand{\epsilon}{\varepsilon}
\newcommand{\eps}{\epsilon}

% Alphabet
\newcommand{\cA}{\mathcal{A}}
\newcommand{\cC}{\mathcal{C}}
\newcommand{\cL}{\mathcal{L}}
\newcommand{\cN}{\mathcal{N}}
\newcommand{\cQ}{\mathcal{Q}}


% Macros for probability
\DeclareMathOperator{\E}{\mathbb{E}} % Expectation
\DeclareMathOperator{\pE}{\widetilde{\mathbb{E}}} % Pseudoexpectation

% Macros for optimization
\newcommand{\OPT}{\text{OPT}}
\newcommand{\SoS}{\text{SoS}}

% Macro for indicator function
\newcommand{\Ind}[1]{\mathbf{1}\left ( #1\right )}

% Title and author information
\title{Problem Set 1}
\author{Samuel B. Hopkins}
\date{\today}

% Document starts here
\begin{document}

\maketitle

Due: 9/24, 11:59pm.

Please typeset your solutions in LaTeX.

\begin{problem}[SoS proofs beyond eigenvalues]
We saw in lecture that if $M \in \mathbb{R}^{n \times n}$ is a symmetric matrix with maximum eigenvalue $\lambda_{max}$, then there is always a degree-2 SoS proof that $x^\top M x \leq \lambda_{max} \cdot \|x\|^2$ -- that is, the polynomial $\lambda_{max} \cdot \|x\|^2 - x^\top M x$ is a sum of squares.

\begin{enumerate}
\item Show that this bound is tight, in the sense that if $c$ is such that $c \|x\|^2 - x^\top M x$ is a sum of squares, then $c \geq \lambda_{max}$.

\item Construct a symmetric matrix $M$ such that there exists $c < \lambda_{max}(M)$ and linear functions $f_1,\ldots,f_m$ such that $c \cdot \|x\|^2 - x^\top M x = \sum_{i \leq m} f_i(x)^2$ for every $x \in \{ \pm 1\}^n$. This shows that the flexibility in a quadratic proof to use a sum of squares polynomial which is equal to $c \|x\|^2 - x^\top M x$ only for certain $x$s (namely, $x \in \{\pm 1\}^n$) makes the definition more powerful.
\end{enumerate}
\end{problem}


\begin{problem}[Max cut in almost-bipartite graphs]
Show that there is a polynomial-time algorithm with the following guarantee: given a graph $G = (V,E)$ such that there is a cut which cuts $(1-\epsilon)|E|$ edges, the algorithm outputs a cut which cuts $(1-\tilde{O}(\sqrt{\epsilon}))|E|$ edges. $(\tilde{O}$ can hide factors of $\log(1/\epsilon)$, though this is not strictly necessary.)

You may use the following basic anticoncentration fact for Gaussians: if $Z \sim N(0,1)$, then $Pr(|Z| \leq \delta) = O(\delta)$.
\end{problem}

\begin{problem}[Cauchy-Schwarz for Pseudoexpectations]
An important fact about any probability distribution $\mu$ is that for any real-valued $f$ and $g$, $\mathbb{E}_{x \sim \mu} f(x) g(x) \leq \sqrt{\mathbb{E} f(x)^2} \cdot \sqrt{ \mathbb{E} g(x)^2 }$. Show that if $\tilde{\mathbb{E}}$ is a (degree 2) pseudoexpectation and $f,g$ are linear functions, one has $\tilde{\mathbb{E}} f(x) g(x) \leq \sqrt{\tilde{\mathbb{E}} f(x)^2} \cdot \sqrt{\tilde{\mathbb{E}} g(x)^2}$.
\end{problem}

\begin{problem}[Max cut on the triangle]
  The three-edge triangle graph has a max-cut value of $2$.
  We saw in class that there is a quadratic proof that the maximum cut is at most $2.9$.
  Is there a quadratic proof that the max-cut value is at most $2.00001$?
  (Prove the correctness of your answer.)
\end{problem}




\end{document}
