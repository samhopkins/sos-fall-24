\documentclass[11pt]{article}

% Packages for math. MUST LOAD BEFORE FONT PACKAGES
\usepackage{amsmath, amssymb, amsthm}


% Set font to Palatino
\usepackage{newpxmath}
\usepackage{newpxtext}
\linespread{1.05}         % Palatino needs more leading (space between lines)
\usepackage[T1]{fontenc}

\usepackage{microtype}
\usepackage{url}
\usepackage{tikz}
\usetikzlibrary{patterns}

\usepackage{color}
\usepackage{tcolorbox}



\usepackage[colorlinks=true,urlcolor=blue,linkcolor=blue]{hyperref}


% Page margins
\usepackage[margin=1in]{geometry}


% Theorem environments
\newtheorem{theorem}{Theorem}
\newtheorem{lemma}[theorem]{Lemma}
\newtheorem{proposition}[theorem]{Proposition}
\newtheorem{corollary}[theorem]{Corollary}
\newtheorem{fact}[theorem]{Fact}
\newtheorem{definition}[theorem]{Definition}
\newtheorem{example}[theorem]{Example}
\newtheorem{remark}[theorem]{Remark}
\theoremstyle{definition}
\newtheorem{problem}[theorem]{Problem}
\newtheorem*{question}{Question}
\newtheorem{exercise}[theorem]{Exercise}

% Delimiter macros
\newcommand{\paren}[1]{\left( #1 \right)}
\newcommand{\brac}[1]{\left[ #1 \right]}
\newcommand{\iprod}[1]{\langle #1 \rangle}
\newcommand{\abs}[1]{\left| #1 \right|}
\newcommand{\norm}[1]{\left\| #1 \right\|}

% Macros for real and natural numbers
\newcommand{\R}{\mathbb{R}} % Real numbers
\newcommand{\N}{\mathbb{N}} % Natural numbers

\renewcommand{\epsilon}{\varepsilon}
\newcommand{\eps}{\epsilon}

% Alphabet
\newcommand{\cA}{\mathcal{A}}
\newcommand{\cC}{\mathcal{C}}
\newcommand{\cL}{\mathcal{L}}
\newcommand{\cN}{\mathcal{N}}
\newcommand{\cQ}{\mathcal{Q}}


% Macros for probability
\DeclareMathOperator{\E}{\mathbb{E}} % Expectation
\DeclareMathOperator{\pE}{\widetilde{\mathbb{E}}} % Pseudoexpectation

% Macros for optimization
\newcommand{\OPT}{\text{OPT}}
\newcommand{\SoS}{\text{SoS}}

% Macro for indicator function
\newcommand{\Ind}[1]{\mathbf{1}\left ( #1\right )}

% Title and author information
\title{Problem Set 2}
\author{Samuel B. Hopkins}
\date{Last updated \today}

% Document starts here
\begin{document}

\maketitle

Due: 10/8, 11:59pm.

Please typeset your solutions in LaTeX.

\begin{problem}[On $\vDash$, borrowed from Aaron Potechin]

Consider the following polynomial equation in $3$ variables, $x,y,z$.
\[ (x^2 + 1) y = z^2. \]
Because it implies $y = \frac{z^2}{x^2+1}$, any solution $(x,y,z)$ to the above must have $y \ge 0$. We will see if sum-of-squares can capture this reasoning.

\begin{enumerate}
  \item Construct a degree $4$ pseudoexpectation $\pE$ in variables $x,y,z$ such that $\pE \vDash (x^2+1)y = z^2$ but $\pE y < 0$. (Computer-aided proofs are allowed.)

  By $\pE \vDash (x^2+1)y = z^2$, we mean that for any polynomial $p$ of degree at most $1$ in $x,y,z$, $\pE p(x,y,z)(x^2+1)y = \pE p(x,y,z)z^2$.

  \item Despite the above, show that there exists a sum-of-squares refutation to the following system of polynomial inequalities, for any $c > 0$: $\{ (x^2+1)y = z^2 , y \le -c \}$.
\end{enumerate}
\end{problem}

\begin{problem}
  Suppose $\pE$ is a pseudoexpectation of degree $d$, with $d$ even, and $\pE \vDash p \leq 0, p \geq 0$ for some polynomial $p$. (Informally, we have been writing $\pE \vDash p = 0$.)
  Show that for every $q$ such that the degree of $pq$ is at most $d$, we have $\pE pq = 0$.
\end{problem}

\begin{problem}
  Unreleased.
\end{problem}

\end{document}
