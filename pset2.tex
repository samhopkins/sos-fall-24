\documentclass[11pt]{article}

% Packages for math. MUST LOAD BEFORE FONT PACKAGES
\usepackage{amsmath, amssymb, amsthm}


% Set font to Palatino
\usepackage{newpxmath}
\usepackage{newpxtext}
\linespread{1.05}         % Palatino needs more leading (space between lines)
\usepackage[T1]{fontenc}

\usepackage{microtype}
\usepackage{url}
\usepackage{tikz}
\usetikzlibrary{patterns}

\usepackage{color}
\usepackage{tcolorbox}



\usepackage[colorlinks=true,urlcolor=blue,linkcolor=blue]{hyperref}


% Page margins
\usepackage[margin=1in]{geometry}


% Theorem environments
\newtheorem{theorem}{Theorem}
\newtheorem*{theorem*}{Theorem}
\newtheorem{lemma}[theorem]{Lemma}
\newtheorem{proposition}[theorem]{Proposition}
\newtheorem{corollary}[theorem]{Corollary}
\newtheorem{fact}[theorem]{Fact}
\newtheorem{definition}[theorem]{Definition}
\newtheorem{example}[theorem]{Example}
\newtheorem{remark}[theorem]{Remark}
\theoremstyle{definition}
\newtheorem{problem}[theorem]{Problem}
\newtheorem{bproblem}[theorem]{Bonus Problem}
\newtheorem*{question}{Question}
\newtheorem{exercise}[theorem]{Exercise}

% Delimiter macros
\newcommand{\paren}[1]{\left( #1 \right)}
\newcommand{\brac}[1]{\left[ #1 \right]}
\newcommand{\iprod}[1]{\langle #1 \rangle}
\newcommand{\abs}[1]{\left| #1 \right|}
\newcommand{\norm}[1]{\left\| #1 \right\|}
\newcommand{\poly}{\mathrm{poly}}

% Macros for real and natural numbers
\newcommand{\R}{\mathbb{R}} % Real numbers
\newcommand{\N}{\mathbb{N}} % Natural numbers

\renewcommand{\epsilon}{\varepsilon}
\newcommand{\eps}{\epsilon}

% Alphabet
\newcommand{\cA}{\mathcal{A}}
\newcommand{\cC}{\mathcal{C}}
\newcommand{\cL}{\mathcal{L}}
\newcommand{\cN}{\mathcal{N}}
\newcommand{\cQ}{\mathcal{Q}}


% Macros for probability
\DeclareMathOperator{\E}{\mathbb{E}} % Expectation
\DeclareMathOperator{\pE}{\widetilde{\mathbb{E}}} % Pseudoexpectation
\newcommand{\Var}{\mathbf{Var}}

% Macros for optimization
\newcommand{\OPT}{\mathsf{OPT}}
\newcommand{\SoS}{\text{SoS}}

% Macro for indicator function
\newcommand{\Ind}[1]{\mathbf{1}\left ( #1\right )}

% Title and author information
\title{Problem Set 2}
\author{Samuel B. Hopkins}
\date{Last updated \today}

% Document starts here
\begin{document}

\maketitle

Due: 10/8, 11:59pm.

Please typeset your solutions in LaTeX.

\begin{problem}[On $\vDash$, borrowed from Aaron Potechin]

Consider the following polynomial equation in $3$ variables, $x,y,z$.
\[ (x^2 + 1) y = z^2. \]
Because it implies $y = \frac{z^2}{x^2+1}$, any solution $(x,y,z)$ to the above must have $y \ge 0$. We will see if sum-of-squares can capture this reasoning.

\begin{enumerate}
  \item Construct a degree $4$ pseudoexpectation $\pE$ in variables $x,y,z$ such that $\pE \vDash (x^2+1)y = z^2$ but $\pE y < 0$. (Computer-aided proofs are allowed.)

  By $\pE \vDash (x^2+1)y = z^2$, we mean that for any polynomial $p$ of degree at most $1$ in $x,y,z$, $\pE p(x,y,z)(x^2+1)y = \pE p(x,y,z)z^2$.

  \item Despite the above, show that there exists a sum-of-squares refutation to the following system of polynomial inequalities, for any $c > 0$: $\{ (x^2+1)y = z^2 , y \le -c \}$.
\end{enumerate}
\end{problem}

\begin{problem}
  Suppose $\pE$ is a pseudoexpectation of degree $d$, with $d$ even, and $\pE \vDash p \leq 0, p \geq 0$ for some polynomial $p$. (Informally, we have been writing $\pE \vDash p = 0$.)
  Show that if $p$ has even degree, for every $q$ such that the degree of $pq$ is at most $d$, we have $\pE pq = 0$. Similarly, show that if $p$ has odd degree, for every $q$ such that the degree of $pq$ is at most $d-1$, we have $\pE pq = 0$.
\end{problem}


\begin{problem}
  In class, we saw how Gaussian rounding and global correlation could be used to approximate the max-cut of a graph. In this exercise, we will see how similar ideas can be used for \emph{max-bisection}. Let $G = (V,E)$ be a graph with $|V|=n$ even. The goal in the max-bisection problem is to determine
  \[ \OPT = \max_{\substack{S \subseteq V \\ |S| = n/2}} E(S,\overline{S}), \]
  where $E(S,\overline{S})$ is the size of the cut corresponding to $S$, that is, the number of edges between $S$ and $\overline{S}$. The goal in this exercise will be to prove the following theorem.

  \begin{theorem*}
    Let $G$ be a regular graph with max-bisection value at least $(1-\eps) |E|$. There exists an algorithm running in time $n^{(1/\eps)^{O(1)}}$ that outputs a bisection cutting $(1-O(\sqrt{\eps})) |E|$ edges.
  \end{theorem*}

  Let $\pE$ be a pseudodistribution over $\{\pm 1\}^n$ such that
  \[ \pE \vDash \left\{ \frac{1}{4} \sum_{ij \in E} (y_i-y_j)^2 \ge (1 - \kappa) |E| , \sum y_i = 0 \right\}. \]

  \begin{enumerate}
    % \setcounter{enumi}{1}
    \item Suppose we apply Gaussian rounding to $\pE$ to produce a random vector $z \in \{\pm 1\}^n$. Show that if the global information of $\pE$ is at most $\delta$, then $\Var \left(\sum z_i\right) \le \delta^{\Omega(1)} \cdot n^2$.

    \item For $\delta > 0$, explain how to round $\pE$ of degree $\poly(1/\delta)$ sufficiently large to a distribution $z$ over $\{\pm 1\}^n$ such that
    \[ \frac{1}{4} \E \sum_{ij \in E} (z_i - z_j)^2 \ge (1 - O(\sqrt{\kappa}))|E| \]
    and
    \[ \Var \left( \sum z_i \right) \le \delta n^2. \]

    \item Using the above, design a (randomized) algorithm running in time $n^{(1/\eps)^{O(1)}}$ that outputs $z \in \{\pm 1\}^n$ such that $\sum z_i = 0$ and
    \[ \frac{1}{4} \sum_{ij \in E} (z_i - z_j)^2 \ge (1-O(\sqrt{\eps})) |E|. \]
    Conclude that you have proved the theorem.
  \end{enumerate}
\end{problem}

\begin{bproblem}[Integrality gaps for max-cut, borrowed from Pravesh Kothari]
  Let $C_n$ be the cycle graph on vertex set $[n]$ with edge set $E$. Further suppose that $n$ is odd.
  The size of the max-cut in $C_n$ is $n-1$. Recall from your solution to Problem 2 of the first problem set that this implies that for any degree $2$ pseudoexpectation $\pE$ on $\{\pm 1\}^n$, $\pE\left[ \frac{1}{4} \sum_{ij \in E} (x_i - x_j)^2 \right] \le \left(1 - O\left( \frac{1}{n^2} \right)\right) n$. We will start by seeing that this is tight for degree $2$ pseudoexpectations.

  Let $L$ the Laplacian of $C_n$ defined by $L_{ii} = 2$ for each $i$, and $L_{ij}$ is $-1$ if $ij$ is an edge and $0$ otherwise. Observe that for $x \in \{\pm 1\}^n$, the size of the cut associated to $x$ is equal to $\frac{1}{4} \cdot x^\top L x$.

  For each $0 \le k \le n/2$, let $x_k,y_k$ be vectors with coordinates $(x_k)_i = \cos(2\pi ik/n)$ and $(y_k)_i = \sin(2\pi ik/n)$.


  \begin{enumerate}
    \item Prove that $x_k$ and $y_k$ are eigenvectors of $L$ with eigenvalues $2 - 2 \cos(2\pi k/n)$.
    \item Prove that the diagonal entries of the matrix $M_k = x_k x_k^\top + y_k y_k^\top$ are $1$.
    \item Prove that there is a degree $2$ pseudoexpectation $\pE_k$ on $\{\pm1\}^n$ with $\pE x = 0$ and $\pE xx^\top = M_k$. Using this, prove that for $k = \frac{n-1}{2}$, $\pE \left[ \frac{1}{4} \sum_{ij \in E} (x_i - x_j)^2 \right] \ge \left( 1 - O\left( \frac{1}{n^2} \right) \right) n$.
  \end{enumerate}

  Next, we will see that degree $6$ pseudoexpectations do not face such barriers (for the cycle graph).

  \begin{enumerate}
    \setcounter{enumi}{3}
    \item Prove that for degree $6$ pseudoexpectations $\pE$ over $\{\pm 1\}^n$, the squared triangle inequality holds: $\pE (x_i - x_j)^2 \le \pE (x_i - x_k)^2 + \pE (x_k - x_j)^2$. For a harder exercise, prove this for degree $4$ pseudoexpectations.
    \item Prove that for any degree $6$ pseudoexpectation $\pE$, $\pE \left[ \frac{1}{4} \sum_{ij \in E} (x_i - x_j)^2 \right] \le n-1$.
  \end{enumerate}

\end{bproblem}

\end{document}