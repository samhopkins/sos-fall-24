\documentclass[11pt]{article}

% Packages for math. MUST LOAD BEFORE FONT PACKAGES
\usepackage{amsmath, amssymb, amsthm}


% Set font to Palatino
\usepackage{newpxmath}
\usepackage{newpxtext}
\linespread{1.05}         % Palatino needs more leading (space between lines)
\usepackage[T1]{fontenc}

\usepackage{microtype}
\usepackage{url}
\usepackage{tikz}
\usetikzlibrary{patterns}

\usepackage{color}
\usepackage{tcolorbox}



\usepackage[colorlinks=true,urlcolor=blue,linkcolor=blue]{hyperref}


% Page margins
\usepackage[margin=1in]{geometry}


% Theorem environments
\newtheorem{theorem}{Theorem}
\newtheorem{lemma}[theorem]{Lemma}
\newtheorem{proposition}[theorem]{Proposition}
\newtheorem{corollary}[theorem]{Corollary}
\newtheorem{fact}[theorem]{Fact}
\newtheorem{definition}[theorem]{Definition}
\newtheorem{example}[theorem]{Example}
\newtheorem{remark}[theorem]{Remark}
\theoremstyle{definition}
\newtheorem{problem}[theorem]{Problem}
\newtheorem*{question}{Question}
\newtheorem{exercise}[theorem]{Exercise}

% Delimiter macros
\newcommand{\paren}[1]{\left( #1 \right)}
\newcommand{\brac}[1]{\left[ #1 \right]}
\newcommand{\iprod}[1]{\langle #1 \rangle}
\newcommand{\abs}[1]{\left| #1 \right|}
\newcommand{\norm}[1]{\left\| #1 \right\|}

% Macros for real and natural numbers
\newcommand{\R}{\mathbb{R}} % Real numbers
\newcommand{\N}{\mathbb{N}} % Natural numbers

\renewcommand{\epsilon}{\varepsilon}
\newcommand{\eps}{\epsilon}

% Alphabet
\newcommand{\cA}{\mathcal{A}}
\newcommand{\cC}{\mathcal{C}}
\newcommand{\cL}{\mathcal{L}}
\newcommand{\cN}{\mathcal{N}}
\newcommand{\cQ}{\mathcal{Q}}


% Macros for probability
\DeclareMathOperator{\E}{\mathbb{E}} % Expectation
\DeclareMathOperator{\pE}{\widetilde{\mathbb{E}}} % Pseudoexpectation

% Macros for optimization
\newcommand{\OPT}{\text{OPT}}
\newcommand{\SoS}{\text{SoS}}

% Macro for indicator function
\newcommand{\Ind}[1]{\mathbf{1}\left ( #1\right )}

% Title and author information
\title{Peer Review Instructions}
\author{Samuel B. Hopkins}
\date{\today}

% Document starts here
\begin{document}

\maketitle

In addition to solving problem sets, you will provide peer reviews of solutions written by other students.
You will thereby get to practice the crucial skill of giving feedback on technical writing to help authors clarify their meaning.
The quality of peer feedback you provide will contribute to your final course grade.
I will provide you with reference solutions to use when giving feedback.
Peer reviews are due one week after the deadline of each problem set.
They will be assigned and completed via Canvas.

\paragraph{Comments}
Most importantly, you will provide written feedback in your peer review.
This is an opportunity to help your peers improve their technical mathematics and their writing.
(And, you may have some realizations about your own solutions in the process.)
If an equation is incorrect, say so!
If a sentence or paragraph is confusingly-worded, flag this as well.
Please be kind but honest in your critiques.


\paragraph{Numerical scores}
For each problem on a problem set, you will assign a score between $0$ and $3$ (inclusive).
These scores will inform my final grading at the end of the semester.
As a reminder, courses at MIT are not graded on a curve!
This means that giving your classmates' solutions lower scores than they deserve is not to your advantage.

The scores have the following meanings:

A score of $3$ indicates near-perfection: the solution is technically correct, and exposition is as clear or clearer than the reference solution.
This solution would be suitable as the proof of a lemma in a mathematics or computer science journal or conference publication.

A score of $2$ indicates correctness, perhaps with a missing detail or two, and exposition clear enough to be understood.
This solution would raise eyebrows from a reviewer for a mathematics or computer science journal or conference, but would not prompt serious concerns about correctness.

A score of $1$ indicates that some attempt was made at the problem, but the solution is either technically incorrect or insufficiently clearly exposited to be understood.
This solution would prompt serious concerns about correctness, if it appeared in a journal or conference submission.

A score of $0$ indicates that no attempt was made at the problem.


\end{document}
